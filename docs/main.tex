\documentclass{article}
\usepackage[utf8]{inputenc} 
% \usepackage[swedish]{babel} 
\usepackage{amsmath, amsfonts, amssymb, amsthm}
\usepackage{siunitx}
\usepackage{geometry} 
\usepackage{graphicx} 
\usepackage{hyperref} 
\usepackage{enumerate}
\usepackage{booktabs} 
\usepackage{array}
\usepackage{tikz}
\usepackage{tcolorbox}
\usepackage{pgfplots}
\usepackage{listings}
\pgfplotsset{compat=newest}

\geometry{a4paper, margin=1in}


\newtheorem{proposition}{Påstående}
\renewcommand{\qedsymbol}{QEB}
\tcolorboxenvironment{proposition}{
  boxrule=0pt,
  boxsep=2mm,
  colback={black!10}, % Light orange background
  colframe={black!50}, % Darker orange frame
  coltitle=blacf(x),
  enhanced,
  sharp corners,
  borderline west={2mm}{0pt}{orange}
}

\title{Ma5 AO5 Omfångsrika problem}
\author{Eduards Abisevs \\ Tetek-21}
\date{\today}

\begin{document}

\maketitle

\section*{Problemformulering}
I en befolkningsmodell för en stor stad kan befolkningstätheten $x$ km från
centrum approximeras med funktionen 
\[
f(x) = \frac{95\,000}{x^2 + 10x + 16}
\]
där $y$ är antalet människor per kvadratkilometer.

Uppgiften handlar om att beräkna hur många människor som bor inom en radie från
centrum. Dvs att teckna ett radial area uttryck, given en täthetsfunktion
beräkna befolkningen vid givna gränser \( a \) och \( b \).

% \noindent
% \subsection*{Givet}
% Av uppgiften ett ledtråd. att teckna ett uttryck.

\noindent
\subsection*{Sökt}
\begin{enumerate}[(a)]
    \item Uppskatta hur många människor som bor inom en radie 5 km från centrum.
    \item Hur många människor bor mellan 5 och 10 km från centrum.
\end{enumerate}

\begin{proposition}[Befolkningsmodell inom en radie från centrum] Given nedre
	gräns \( a \) och övre gräns \( b \) i km, där \( 0 \le a < b \le 10 \) och
	\( a, b \in \mathbb{R}^+ \) därmed gäller modellen: \begin{align*}
		g(a,b) &= \int_{a}^{b} \left(f(x) \cdot 2\pi x\right) \, dx
		= \int_{a}^{b} \frac{95\,000 \cdot 2\pi x}{x^2 + 10x + 16} \, dx
		  \\
		&= \frac{190\,000 \pi}{3}
		\left(4\ln{\left(\frac{b + 8}{a + 8}\right)} 
			-
		\ln{\left(\frac{b + 2}{a + 2}\right)} \right)
	\end{align*}
	Motivering av \( b_\max \, = 10 \) är av att i uppgiten är det inte givet
	i vilken interval \( f(x) \) ger godtycklig approximation, därmed togs
valet att ha gränsen vid högsta efterfrågade värden som är \( 10 \) km.
\end{proposition}

\begin{proof}
För att bevisa att funktionen \( g(a,b) \) gäller för godtyckliga \( a \) och \(
b \) ska integrerings processen härledas, samt resultatet av godtyckliga tal
jämföras med approxomitiv lössning med hjälp av digitala hjälp medel (Python
program).

Ett metod att göra detta är att teckna ett uttryck för ett litet cirklesegment för
att beräkna radial area. Sedan multiplicera den area med
befolkningstäthetsfunktionen och som sist summera alla små cirkleringar mellan
gränserna \( a \) och \( b \).

\textbf{Steg 1. Teckna ett uttryck för area av små cirkel segment} \\
	% Circle bild
	\begin{figure}[h]
	\centering
	\label{fig:summa}
	\begin{tikzpicture}
	% Define radii
	\def\rBig{1}
	\def\rSmall{0.5}


	% Draw the outer circle
	\filldraw[color=grey, fill=red!5, very thick] (0,0) circle (\rBig);

	% Draw the inner circle
	\filldraw[color=grey, fill=white, very thick] (0,0) circle (\rSmall);

	% Draw origo
	\filldraw[color=black] (0,0) circle (1.5pt);

	% Draw the segment indicating \Delta x
	\draw[thick] (\rSmall,0) -- (\rBig,0) node[right] {$\Delta x$};
	
	% radie and annotiations
	\draw[thick] (0,0) -- (-1,0) node[left] {$x$};

	\draw [-latex, line width=0.5mm] (3,0) -- (4,0);

	\draw[thick, fill=red!5] (6,-\rSmall/2) rectangle (9,\rSmall/2);
	\node at (7.5, -\rSmall/2 - 0.3) {$2\pi\/x$}; 
	\node at (6 - 0.3, 0) [left] {$\Delta x$};  

	\end{tikzpicture}
	% \caption{Ilustatration hur cirklsegment kan beskrivas som ett rektangle}
	\end{figure}
Låt \( x \) vara avstånd från centrum av en storstad dvs radie och \( \Delta x
\) ett litet avstånd in mot centrum. Matematiskt kan detta sammband beskrivas
som area \( A \),  där \( 2 \pi x \) är cirkel segments omkrets och \( \Delta x
\) tjockleken eller höjden 
\begin{align*}
	A = 2\pi x \cdot \Delta x
\end{align*}

\textbf{Steg 1.1 Teckna ett uttryck av befolkningen} \\
Given täthetsfunktionen \( f(x) \) multiplicerat med radial area uttryck \( A
\), ger detta ihop antal \( M \) människor  i den radiala området \( x \) km från
centrum.
\begin{align*}
	M = f(x) \cdot A
\end{align*}

\textbf{Steg 1.2. Approximation med Riemannsumma} \\
Låt S vara summa av alla människor \( M \) i intervallet från \( a \) till \( b \).
Sedan låt \( \Delta x = \frac{b-a}{n} \) och \( x_i \) var definerat som \( x_i
= a + i \Delta x \) för \( i = 0, 1, 2, \ldots, n \). Därmed Reimannsumma kan
approximera arean under kurvan \( M \):
\begin{align*}
S = \sum_{i=0}^{n} f(x_i) \cdot 2 \pi x_i \cdot \Delta x
\end{align*}
Låt nu \( \Delta x \to 0 \) därefter övergår  Reimannsumma till ett bestämd integral, 
\begin{align*}
	\lim_{\Delta x \to 0} \sum_{i=0}^{\infty} f(x_i) \cdot 2 \pi x_i \cdot \Delta x &=\int_{a}^{b}{\left(f(x) \cdot 2\pi x\right) \, dx } \\
\end{align*}

\textbf{Steg 2. Beräkna bestämd integral} \\ 
Låt nu \( g(a, b) \) vara ett funktion till bestämd integral mellan gränserna
\( a \) och \( b \)
\begin{align*}
	g(a,b) &=\int_{a}^{b}{\left(f(x) \cdot 2\pi x\right) \, dx } \\
	       &= \int_{a}^{b}{\left(\frac{95\, 000 \cdot 2 \pi x}{x^2 + 10x + 16}\right) \, dx } \\
	&= 190\, 000 \pi \int_{a}^{b}{\left(\frac{x}{x^2 + 10x + 16}\right) \, dx }
	&&\text{(Bryta ut konstanterna)} \\
	% Visa steget hur man faktoriserer
	&= 190\, 000 \pi \int_{a}^{b}{\left(\frac{x}{(x + 8)(x + 2)}\right) \, dx }
	&&\text{(Faktorisera nämnaren)} \\
	&= 190\, 000 \pi \int_{a}^{b}{\left(\frac{A}{(x + 8)} + \frac{B}{(x + 2)}\right) \,
	dx }
	&&\text{(Simplifiera genom 2.1)} \\
\end{align*}

\indent
\textbf{Steg 2.1. Partialbråkupdelning}
\begin{align*}
	\frac{x}{(x + 8)(x + 2)} &= \frac{A}{x +8} + \frac{B}{x + 2} \\
				 &= \frac{A(x + 2)}{(x + 8)(x + 2)} + \frac{B(x
				 + 8)}{(x + 8)(x + 2)}
				 &&\text{(Förläng bråk)}\\
				 &= \frac{Ax + 2A + Bx + 8B}{(x + 8)(x + 2)} 
				 &&\text{(Utveckla täljare)}\\
				 &= \frac{x(A + B) + 2A + 8B}{(x + 8)(x + 2)}
				 &&\text{(Faktorisera ut gemensam \( x \) )} \\
\end{align*}
% \indent
För att lösa ut täljaren \( A \) och \( B \) kan en enkelt ekvationssytem
defineras, där ekvationen (i) tillhör till antal \( x \)  i ursrungliga bråkets
täljare och (ii) till konstanterna.
\begin{align*}
	\begin{cases}
		A + B = 1 \\
		2A + 8B = 0 &(ii) 
	\end{cases}
\end{align*}
Ekvationen (i) omvandlas till $A = 1 - B$ och sedan sätts in i ekvationen (ii)
\begin{align*}
	&2A + 8B = 0 \\
	&2(1 - B) + 8B = 0 \\ % &&\text{(Sätt (i) i (ii) )} \\
	&2 - 2B + 8B = 0 \\
	&6B =  -2 \\
	&B = -\frac{1}{3} \\
\end{align*}
Sätter man in B i (i) får man
\begin{align*}
	&A + (-\frac{1}{3}) = 1 \\
	&A = 1 + \frac{1}{3} \\
	&A = \frac{4}{3}
\end{align*}
Lössningen till ekvationssystemet är 
\begin{align*}
	\begin{cases}
		A = \frac{4}{3} \\
		B = - \frac{1}{3}
	\end{cases}
\end{align*}

\textbf{Steg 2.2. Lössningar till \( A \) och \( B \) sätts i integralen}
\begin{align*}
	g(a,b) &= \int_{a}^{b}{\left(f(x) \cdot 2\pi x\right) \, dx }\\
	       & \vdots \\
	&= 190\, 000 \pi \int_{a}^{b}{\left(\frac{A}{(x + 8)} + \frac{B}{(x + 2)}\right) \,
	dx }  \\
	&= 190\, 000 \pi \int_{a}^{b}{\left(\frac{\frac{4}{3}}{(x + 8)} 
		+ \frac{-\frac{1}{3}}{(x + 2)}\right) \,
	dx }  
	&&\text{(Bryt ut i två integraler)}
	\\
	&= 190\, 000 \pi \left(\int_{a}^{b}{\frac{\frac{4}{3}}{(x + 8)} \, dx} 
		- \int_{a}^{b}{{\frac{\frac{1}{3}}{(x + 2)} \,
	dx}\right) }
	&&\text{(bryt ut \( -1 \) från andra integral)} \\
\end{align*}

\textbf{Steg 2.3. Beräkna första integralen}
\begin{align*}
	\int_a^b \frac{\frac{4}{3}}{x + 8} \, dx 
	&\Rightarrow \int_a^b \frac{a}{t} \, dt = \left[a\ln(|t|)\right]_a^b
	&&\text{där  \( a = \frac{4}{3} \) och \(t = x + 8 \) } \\
	&\Rightarrow \left[\frac{4}{3} \cdot \ln{(|x + 8|)}\right]_a^b 
	&&\text{(Sätt tilbaba \( a,t \))} \\
	&= \frac{4}{3} \cdot \ln{(|b + 8|)} 
	-  \frac{4}{3} \cdot \ln{(|a + 8|)}
	&& \text{(Sätt integrands gränser)} \\
	&= \frac{4}{3} \cdot \left(\ln{(|b + 8)} - \ln{(|a + 8|)} \right) 
	&& \left( \text{Bryt ut } \frac{4}{3} \right) \\
	&\Rightarrow \ln(a) - \ln(b) = \ln{\left(\frac{a}{b}\right)} 
	&&\text{(Logoritmiska egenskapen)} \\
	&\Rightarrow \frac{4}{3} \cdot \ln{\left(\frac{|b + 8|}{|a + 8|}\right)} 
	&&\text{(Simplifiera genom egenskapen)} \\ 
	&=  \frac{4\ln{\left(\frac{|b + 8|}{|a + 8|}\right)}}{3}
	&& \text{(Simplifiera genom multiplikation)} \\
\end{align*}
För att \( \ln(|t|) \) är definerat \( t > 0 \), används absolutbelopp
men för att \( a, b \in \mathbb{R}^+ \) kan
absolutbelopp utelämnas därmed gäller:
\[
	\int_a^b \frac{\frac{4}{3}}{x + 8} \, dx 
	= \frac{4\ln\left(\frac{b + 8}{a + 8}\right)}{3}
\]

\textbf{Steg 2.4. Beräkna andra integralen} \\
Andra integralen beräknas på exakt samma sätt som första för att den är av

samma sorts.
\begin{align*}
	\int_a^b \frac{\frac{1}{3}}{x + 2} \, dx 
	&= \left[\frac{1}{3} \cdot \ln{(x + 2)}\right]_a^b  \\
	&= \frac{\ln{(b + 2)}}{3} - \frac{\ln{(a + 2)}}{3} \\
	&= \frac{\ln{(b + 2)} - \ln{(|a + 2|)}}{3} \\
	&= \frac{\ln\left({\frac{{b + 2}}{{a + 2}}}\right)}{3} \\
\end{align*}

\textbf{Steg 2.5. Sätt beräknade integralerna tillbaka}
\begin{align*}
	g(a,b) &= \int_{a}^{b}{\left(f(x) \cdot 2\pi x\right) \, dx } \\
	       &\vdots \\
	       &= 190\,000\pi \left( 
\frac{4\ln{\left(\frac{b + 8}{a + 8}\right)}}{3}
- \frac{\ln\left({\frac{{b + 2}}{{a + 2}}}\right)}{3}
	       \right) 
	       &&\text{(Sätt in 2.3. och 2.4.)}
	       \\
	       &= 190\,000\pi \left( 
		       \frac{4\ln{\left(\frac{b + 8}{a + 8}\right)}
		       - \ln\left({\frac{b + 2}{a + 2}}\right)}
		       {3} 
	       \right) 
	       &&\text{(Simplifiera under ett bråk streck)} \\
	       &= 190\,000\pi \cdot
		       \frac{1}{3}  
	       \left( 
		       4\ln{\left(\frac{b + 8}{a + 8}\right)}
		       - \ln\left({\frac{b + 2}{a + 2}}\right)
	       \right)
	       &&\left( \text{Faktorisera ut \( \frac{1}{3} \)} \right) \\
	       &= \frac{190\,000\pi}
		       {3}  
	       \left( 
		       4\ln{\left(\frac{b + 8}{a + 8}\right)}
		       - \ln\left({\frac{b + 2}{a + 2}}\right)
	       \right)
	       &&\text{(Simplifiera bråk)} \\
\end{align*}

\textbf{Steg 3. Verifiering av Påstånde 1.} \\
3 Tester utfördes för att jämföra approximativa svaret med svaret som \( g(a,
b) \) ger. I approximativa kaklyleringen med \( \Delta x = 0.00001 \) där 
dator utförde kaklyleringen av \( S \), se Appendix \ref{app:python_code} för källkod.
\begin{center}
	\begin{tabular}{c|c|c|c}
		$a$ & $b$ & $S$ & $g(a, b)$ \\
		\hline\hline
		0 & 1 & 13065.7 & 13065.6 \\
		\hline
		4 & 6 & 65444.9 & 65444.5 \\
		\hline
		9 & 10 & 28178.5 & 28178.2  \\
	\end{tabular}
\end{center}
\end{proof}

\newpage
\section*{Lösning}
\subsection*{Del (a)}
Uppskatta hur många människor som bor inom en radie 5 km från centrum.
För att lösa problemet tillämpas funktion \( g(a,b) \) från Påstånde 1, där \( a
= 0 \) d.v.s. nedre gräns börjar i centrum i storstad och \( b = 5 \).  
\begin{align*}
	g(a,b) &= \frac{190\,000 \pi}{3}
		\left(4\ln{\left(\frac{b + 8}{a + 8}\right)} 
		- \ln{\left(\frac{b + 2}{a + 2}\right)} \right)
		\\ 
	g(0,5) &= \frac{190\,000\pi}{3} \left(4\ln{\left(\frac{5 + 8}{0 + 8}\right)} 
		- \ln{\left(\frac{5 + 2}{0 + 2}\right)}\right) \\
	       &= \frac{760\,000\pi \cdot \ln{\left(\frac{13}{8}\right)}}{3} 
	       - \frac{190\,000\pi \cdot \ln{\left(\frac{7}{2}\right)}}{3} 
	       &&\text{(Exakta svaret)} \\
	       &\approx 137\,142 \\
	       &\approx 137\,000 \, \text{(Människor)}
\end{align*}

\subsection*{Del (b)}
Hur många människor bor mellan 5 och 10 km från centrum. För att beräkna detta
tillämpas funktionen \( g(a,b) \) från Påstånde 1, där \( a = 5 \) och \( b =
10 \).
\begin{align*}
	g(a,b) &= \frac{190\,000 \pi}{3}
		\left(4\ln{\left(\frac{b + 8}{a + 8}\right)} 
		- \ln{\left(\frac{b + 2}{a + 2}\right)} \right)
		\\ 
	g(5,10) &= \frac{190\,000\pi}{3} \left(4\ln{\left(\frac{10 + 8}{5 + 8}\right)} 
		- \ln{\left(\frac{10 + 2}{5 + 2}\right)}\right) \\
	       &= \frac{760\,000\pi \cdot \ln{\left(\frac{18}{13}\right)}}{3} 
	       - \frac{190\,000\pi \cdot \ln{\left(\frac{12}{7}\right)}}{3} \\
	       &\approx 151\,751 \\ 
	       &\approx 152\,000 \, \text{(Människor)}
\end{align*}

\section*{Svar}
\textbf{Del (a)}: $g(0, 5) \approx 137\,000 \, \text{(Människor)}$ \\
\textbf{Del (b)}: $g(5, 10) \approx 152\,000 \, \text{(Människor)}$

\section*{Diskussion av resultatet}
En storstad enligt Wikipedia defineras som stad med fler än \( 100\,000 \)
invånare. Föreställar man sig ett storstad därefeter svaret på (a) och (b) kan
vara rimliga. Ja, de är nära varandra men arean mellan 5 och 10 km är 3 gånger
större än mellan 0 och 5 km. Men tätheten är större mellan 0 och 5 km jämfört
med 5 till 10 km.
\begin{align*}
	A_{\text{0-5}} &= 2\pi r^2	\Rightarrow 2\pi 5^2 = 50\pi
\, \left( \si{\kilo\meter}^2 \right) \\
	A_{\text{5-10}} &= 2\pi 10^2 - 2\pi 5^2 \\
			&= 2\pi(100 - 25) \\
			&= 150\pi \, \left( \si{\kilo\meter}^2 \right)
\end{align*}

\newpage
\appendix
\section{Python kod}
\label{app:python_code}
\lstinputlisting[language=Python, firstline=2, lastline=40]{../simulations/approx.py}


\end{document}
